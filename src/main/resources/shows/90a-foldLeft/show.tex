\documentclass[xcolor=dvipsnames]{beamer}
%\usepackage[OT4]{fontenc}
%\usepackage[utf8]{inputenc}
%\usepackage[polish]{babel}
\usepackage{fancyvrb}
\usepackage{thumbpdf}
\usepackage{relsize}

%\usetheme[left]{Marburg}
%\usetheme{Hannower}
\useinnertheme[shadow]{rounded}
\useoutertheme[right,width=2cm,hideothersubsections]{sidebar}
%\usecolortheme{albatross}
\usecolortheme{crane}
%\setbeamercolor{structure}{fg=OliveGreen!50!black}
%\usecolortheme[named=OliveGreen]{structure}
\usefonttheme{default}
\setbeamercovered{transparent}
\title{Scala -- beauty of foldLeft}
\author{Jacek~Bilski}
\date{\today}
\subject{Scala -- beauty of foldLeft}

\setbeamertemplate{navigation symbols}
{
	\usebeamercolor[fg]{navigation symbols dimmed}
	{
		\insertframenumber\,/\,\inserttotalframenumber
	}
}

\begin{document}

\begin{frame}
\titlepage
\end{frame}

\begin{frame}
\frametitle{Agenda}
\tableofcontents[pausesections]
\end{frame}


\section{Some introduction}

\begin{frame}
\frametitle{Scala}
\begin{itemize}
\item \href{http://www.scala-lang.org/}{http://www.scala-lang.org/}
\item General purpose programming language
\item Runs on JVM
\item Object oriented and functional at the same time
\item Type-safe
\item Reason: there's so much more that Java in the World,
\item \ldots{}and I feel functional programming is the future
\end{itemize}
\end{frame}

\begin{frame}
\frametitle{Project Euler}
\begin{itemize}
\item \href{http://projecteuler.net/}{http://projecteuler.net/}
\item Solving real mathematical or algorithmic problems
\item $>=$425 problems already
\item One new every week
\item Reason: I want to be a better programmer
\item ...and it's fun!
\end{itemize}
\end{frame}

\section{Problem}

\begin{frame}
\frametitle{Problem 24 -- Lexicographic permutations}
A permutation is an ordered arrangement of objects. For example, 3124 is one possible permutation of the digits 1, 2, 3 and 4. If all of the permutations are listed numerically or alphabetically, we call it lexicographic order. The lexicographic permutations of 0, 1 and 2 are:

012   021   102   120   201   210

What is the millionth lexicographic permutation of the digits 0, 1, 2, 3, 4, 5, 6, 7, 8 and 9?
\end{frame}

\section{Solution}

\frame[containsverbatim]{
\frametitle{Quick and dirty}
\begin{Verbatim}[obeytabs=true]
val digits = List(0, 1, 2, 3, 4, 5, 6, 7, 8, 9)
val permutationNumber = 1000000
digits.permutations.toList(permutationNumber - 1)
\end{Verbatim}
}

\begin{frame}
\frametitle{Problem: number vs List}
What we got:\\
$List[Int]$\\
What we need:\\
$scala.math.BigInt$
\end{frame}

\begin{frame}
\frametitle{Solution - foldLeft}
$def$ $foldLeft[B](z: B)(f: (B, A) \Rightarrow B): B$\\
Applies a binary operator to a start value and all elements of this list, going left to right.
\begin{itemize}
\item[B] the result type of the binary operator.
\item[z] the start value.
\item[returns]{the result of inserting f between consecutive elements of this list, going left to right with the start value z on the left:\\
$f(\ldots{}f(z, x_1), x_2, \ldots, x_n)$\\
where $x_1, \ldots, x_n$ are the elements of this list.}
\end{itemize}
\end{frame}

\frame[containsverbatim]{
\frametitle{Quick and dirty -- solution}
\begin{Verbatim}[obeytabs=true]
val digits = List(0, 1, 2, 3, 4, 5, 6, 7, 8, 9)
val permutationNumber = 1000000
val permutation = digits.permutations.
	toList(permutationNumber - 1)
permutation.foldLeft(BigInt(0))(10 * _ + _)
\end{Verbatim}
}

\frame[containsverbatim]{
\frametitle{foldLeft explained}
\begin{Verbatim}[obeytabs=true]
def f(b: BigInt, a: Int): BigInt = b + a
permutation.foldLeft(BigInt(0))(f)
\end{Verbatim}
}

\frame[containsverbatim]{
\frametitle{''Proper'' solution (I think\ldots)}
\begin{Verbatim}[obeytabs=true,fontsize=\relsize{-3}]
def howManyPermutations(elements: Number): Number = factorial(elements)
def generateNthPermutation(elements: List[Int], permNumber: Number): List[Int] = {
  if (elements.isEmpty) {
    Nil
  } else {
    if (elements.size == 1) {
      List(elements.head)
    } else {
      val x = howManyPermutations(elements.size - 1)
      val elementAtCurrentPosition = elements((permNumber / x).toInt)
      elementAtCurrentPosition :: generateNthPermutation(
        elements.filter(_ != elementAtCurrentPosition), permNumber % x)
    }
  }
}
generateNthPermutation(digits, permutationNumber - 1).foldLeft(BigInt(0))(10 * _ + _)
\end{Verbatim}
}

\section{Questions?}

\begin{frame}
\frametitle{Questions?}
\begin{center}
\Huge{?}
\end{center}
\end{frame}


\end{document}
