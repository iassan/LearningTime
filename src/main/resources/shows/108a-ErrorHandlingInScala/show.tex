\documentclass[xcolor=dvipsnames]{beamer}
\usepackage{fancyvrb}
\usepackage{thumbpdf}
\usepackage{relsize}
\usepackage{amsmath}

\useinnertheme[shadow]{rounded}
\useoutertheme[right,width=2cm,hideothersubsections]{sidebar}

\definecolor{ZooplusGreen}{RGB}{129,197,46}
\definecolor{ZooplusBlue}{RGB}{4,6,76}

\setbeamercolor{structure}{fg=ZooplusGreen}

\setbeamercolor{palette primary}{fg=ZooplusBlue,bg=ZooplusGreen!70}
\setbeamercolor{palette secondary}{fg=ZooplusBlue,bg=ZooplusGreen!80}
\setbeamercolor{palette tertiary}{fg=ZooplusBlue,bg=ZooplusGreen!90}
\setbeamercolor{palette quaternary}{fg=ZooplusBlue,bg=ZooplusGreen}

\setbeamercolor{titlelike}{parent=palette quaternary}

\setbeamercolor{block title}{fg=ZooplusBlue,bg=ZooplusGreen}
\setbeamercolor{block title alerted}{use=alerted text,fg=ZooplusBlue,bg=alerted text.fg!75!bg}
\setbeamercolor{block title example}{use=example text,fg=ZooplusBlue,bg=example text.fg!75!bg}

\setbeamercolor{block body}{parent=normal text,use=block title,bg=block title.bg!25!bg}
\setbeamercolor{block body alerted}{parent=normal text,use=block title alerted,bg=block title alerted.bg!25!bg}
\setbeamercolor{block body example}{parent=normal text,use=block title example,bg=block title example.bg!25!bg}

\setbeamercolor{sidebar}{bg=ZooplusGreen!70}

\setbeamercolor{palette sidebar primary}{fg=ZooplusBlue}
\setbeamercolor{palette sidebar secondary}{fg=ZooplusBlue!75}
\setbeamercolor{palette sidebar tertiary}{fg=ZooplusBlue!75}
\setbeamercolor{palette sidebar quaternary}{fg=ZooplusBlue}

\setbeamercolor*{separation line}{}
\setbeamercolor*{fine separation line}{}

\logo{\includegraphics[scale=0.25]{../../zooplus_logo.png}}

\usefonttheme{default}
\setbeamercovered{transparent}
\title{Handling errors in Scala}
\author{Jacek~Bilski}
\date{\today}
\subject{Handling errors in Scala}

\setbeamertemplate{navigation symbols}
{
	\usebeamercolor[fg]{navigation symbols dimmed}
	{
		\insertframenumber\,/\,\inserttotalframenumber
	}
}

\begin{document}

\begin{frame}
\titlepage
\end{frame}

\begin{frame}
\frametitle{Agenda}
\tableofcontents[pausesections]
\end{frame}

% Option, Either, Try


\section{Introduction}

\begin{frame}
\frametitle{Errors in Scala}
\begin{itemize}
\item Scala runs on JVM
\item ...hence we have all exceptions from Java world available,
\item ...but who said they're the best ways of handling issues.
\item Scala is also a functional language, and there problems are handled in a different way.
\end{itemize}
\end{frame}

\section{Option}

\frame[containsverbatim]{
\frametitle{Mean function}
\begin{Verbatim}[obeytabs=true,fontsize=\relscale{0.7},tabsize=2]
object ErrorHandling {

	def main(args: Array[String]): Unit = {
		println(mean(Seq(1, 2, 3)))
	}

	def mean(numbers: Seq[BigDecimal]): BigDecimal =
		numbers.sum / numbers.size
}
\end{Verbatim}
}

\frame[containsverbatim]{
\frametitle{Mean function being mean}
\begin{Verbatim}[obeytabs=true,fontsize=\relscale{0.7},tabsize=2]
object ErrorHandling {

	def main(args: Array[String]): Unit = {
		println(mean(Seq()))
	}

	def mean(numbers: Seq[BigDecimal]): BigDecimal =
		numbers.sum / numbers.size
}
\end{Verbatim}
}

\begin{frame}
\frametitle{Option type}
\begin{itemize}
\item We have $Option$ type.
\item Option has two subtypes:
    \begin{itemize}
    \item $Some$ representing data being there
    \item $None$ representing no data (for whatever reason)
    \end{itemize}
\end{itemize}
\end{frame}

\frame[containsverbatim]{
\frametitle{Mean function version 2}
\begin{Verbatim}[obeytabs=true,fontsize=\relscale{0.7},tabsize=2]
object ErrorHandling {

	def main(args: Array[String]): Unit = {
		println(mean(Seq()))
		println(mean(Seq(1, 2, 3)))
	}

	def mean2(numbers: Seq[BigDecimal]): Option[BigDecimal] =
		if (numbers.size == 0)
			None
		else
			Some(numbers.sum / numbers.size)
}
\end{Verbatim}
}

\begin{frame}
\frametitle{How does the Option influence other parts of the system?}
\begin{itemize}
\item Do we now ''pollute'' our whole system with Options?
\item Answer: no!
\end{itemize}
\end{frame}

\frame[containsverbatim]{
\frametitle{Following computation}
\begin{Verbatim}[obeytabs=true,fontsize=\relscale{0.7},tabsize=2]
object ErrorHandling {

	def main(args: Array[String]): Unit = {
		println(three(mean2(Seq())))    // <- compilation error
	}

    def three(x: BigDecimal): BigDecimal = x * 3

	def mean2(numbers: Seq[BigDecimal]): Option[BigDecimal] =
		if (numbers.size == 0)
			None
		else
			Some(numbers.sum / numbers.size)
}
\end{Verbatim}
}

\frame[containsverbatim]{
\frametitle{Following computation --- map}
\begin{Verbatim}[obeytabs=true,fontsize=\relscale{0.7},tabsize=2]
object ErrorHandling {

	def main(args: Array[String]): Unit = {
		println(mean2(Seq(1, 2, 3)) map three)
	}

    def three(x: BigDecimal): BigDecimal = x * 3

	def mean2(numbers: Seq[BigDecimal]): Option[BigDecimal] =
		if (numbers.size == 0)
			None
		else
			Some(numbers.sum / numbers.size)
}
\end{Verbatim}
}

\section{Questions?}

\begin{frame}
\frametitle{Questions?}
\begin{center}
\Huge{?}
\end{center}
\end{frame}


\end{document}
